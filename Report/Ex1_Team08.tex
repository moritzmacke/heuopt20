\documentclass{scrartcl}
\usepackage[shortlabels]{enumitem}
\usepackage[ngerman]{babel}
\usepackage[utf8]{inputenc}
\usepackage{mathtools}
\usepackage{tikz}
\usepackage{xcolor}
\usepackage{soul}
\usepackage{algorithm}
\usepackage[noend]{algpseudocode}
\usetikzlibrary{positioning}
\usetikzlibrary{arrows}

\begin{document}
	\begin{enumerate}
		\item ...
		\item Given is a (possibly not complete) graph $G=(V,E)$ and edge weigths $d_{i,j}$ for $(i,j) \in E$. To make the graph complete, an additional set of edges $E'$ is defined, consisting of all edges that are not alread in $E$, and we set $d_{i,j} = M$ for $(i,j) \in E'$. We want to calculate a value for $M$ so that any solution that requires traversing an edge from $E'$ is worse than all solutions that do not.
		
		A solution must visit $|V| + 1$ vertices (every node once, the starting node twice), so it always consists of $|V|$ distinct edges. A definite upper bound on the objective value of a valid solution can therefore be obtained via $E_{|V|}^+$ and $E_{|V|}^-$, which are the $|V|$ edges with the largest positive and smallest negative weights, respectively.
		\[ U(G) = \max\left( \left| \sum_{(i,j) \in E_{|V|}^+}{d_{i,j}} \right|, \left| \sum_{(i,j) \in E_{|V|}^-}{d_{i,j}} \right| \right) \]
		
		Meanwhile, the lower bound on the objective value of a solution containing an edge with weight $M > 0$ can be found by assuming all other edges have the smallest negative weights possible.
		\[ L(G,M) = M + \sum_{(i,j) \in E_{|V|-1}^-}{d_{i,j}} \]
		
		Now we just need to find $M$ so that the disequality $L(G,M) > U(G)$ is satisfied:
		\begin{align*}
			M + \sum_{(i,j) \in E_{|V|-1}^-}{d_{i,j}} &> \max\left( \left| \sum_{(i,j) \in E_{|V|}^+}{d_{i,j}} \right|, \left| \sum_{(i,j) \in E_{|V|}^-}{d_{i,j}} \right| \right)\\
			M &> \max\left( \left| \sum_{(i,j) \in E_{|V|}^+}{d_{i,j}} \right|, \left| \sum_{(i,j) \in E_{|V|}^-}{d_{i,j}} \right| \right) - \sum_{(i,j) \in E_{|V|-1}^-}{d_{i,j}}
		\end{align*}
	\end{enumerate}
\end{document}